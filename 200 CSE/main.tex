\documentclass[compress,aspectratio=169]{beamer}

\usepackage[english]{babel}
\usepackage[T1]{fontenc}
\usepackage[utf8]{inputenc}
\usepackage{graphicx}
\usepackage{booktabs}
\usepackage{amsmath}
\usepackage{tikz}
\usetikzlibrary{
  trees, shapes, arrows.meta, positioning, calc,
  decorations.pathreplacing, backgrounds, fit, shadows.blur
}
\usetheme{hsrm}

\setbeamercovered{transparent=15}

\colorlet{PRI}{darkRed}
\colorlet{SEC}{logoRed}
\colorlet{HL}{accentRed}
\colorlet{LG}{lightGray}

\tikzset{
  treenode/.style={
    rectangle, draw=PRI, fill=LG!40,
    rounded corners=3pt, minimum width=1.4cm,
    minimum height=0.52cm, font=\small\bfseries,
    blur shadow={shadow blur steps=3, shadow xshift=0.4mm, shadow yshift=-0.4mm}
  },
  leafnode/.style ={treenode, fill=HL!25,  draw=HL!80!black},
  rootnode/.style ={treenode, fill=PRI,    text=white, draw=PRI},
  mbr/.style      ={thick, rounded corners=1pt},
  ghost/.style    ={opacity=0.18},
  alive/.style    ={opacity=1.0},
}

\newcommand{\spatialgrid}[2]{%   #1=xmax  #2=ymax
  \draw[LG!70, very thin, step=1] (0,0) grid (#1,#2);
  \draw[-{Stealth[scale=0.7]}](0,0)--(#1+0.3,0) node[right,font=\tiny]{$x$};
  \draw[-{Stealth[scale=0.7]}](0,0)--(0,#2+0.25) node[above,font=\tiny]{$y$};
}

\title{R-Trees}
\subtitle{A Dynamic Index Structure for Spatial Searching}
\date{February 23, 2026}
\author{}
\institute{}

% =============================================================
\begin{document}
% =============================================================

% ── Custom Title Page ─────────────────────────────────────────
\begin{frame}[plain]
\begin{tikzpicture}[remember picture, overlay]

  %% Dark red background panel (left 60%)
  \fill[PRI]
    (current page.north west) rectangle
    ([xshift=0.60\paperwidth] current page.south west);

  %% Subtle grid texture on red panel
  \begin{scope}
    \clip (current page.north west) rectangle
          ([xshift=0.60\paperwidth] current page.south west);
    \draw[white, opacity=0.06, step=0.4cm]
      (current page.north west) grid
      ([xshift=0.60\paperwidth] current page.south west);
  \end{scope}

  %% Diagonal accent stripe at seam
  \fill[white, opacity=0.07]
    ([xshift=0.60\paperwidth] current page.north west) --
    ([xshift=0.635\paperwidth] current page.north west) --
    ([xshift=0.635\paperwidth] current page.south west) --
    ([xshift=0.60\paperwidth]  current page.south west) -- cycle;

  %% BUET logo (top-left corner, on white bg area)
  \node[anchor=north west, inner sep=0pt] at
    ([xshift=0.625\paperwidth, yshift=-0.04\paperheight] current page.north west)
    {\includegraphics[height=1.0cm]{BUET_LOGO.png}};

  %% Title
  \node[
    anchor=west, text=white,
    font=\LARGE\bfseries,
    text width=0.52\paperwidth, align=left
  ] at ([xshift=0.05\paperwidth, yshift=0.13\paperheight] current page.west)
    {R-Trees};

  %% Subtitle
  \node[
    anchor=west, text=white!80,
    font=\normalsize,
    text width=0.52\paperwidth, align=left
  ] at ([xshift=0.05\paperwidth, yshift=0.04\paperheight] current page.west)
    {A Dynamic Index Structure\\[3pt]for Spatial Searching};

  %% Separator line
  \draw[white, opacity=0.35, line width=0.8pt]
    ([xshift=0.05\paperwidth, yshift=-0.025\paperheight] current page.west) --
    ([xshift=0.55\paperwidth, yshift=-0.025\paperheight] current page.west);

  %% Based-on + date
  \node[
    anchor=west, text=white!55,
    font=\scriptsize\itshape
  ] at ([xshift=0.05\paperwidth, yshift=-0.065\paperheight] current page.west)
    {Based on Antonin Guttman's 1984 Paper \quad$\cdot$\quad February 23, 2026};

  %% Right panel: Team card
  \node[
    anchor=center,
    fill=white, rounded corners=7pt,
    inner sep=14pt,
    drop shadow={shadow xshift=0.5mm, shadow yshift=-0.5mm, opacity=0.15}
  ] at ([xshift=0.805\paperwidth, yshift=0pt] current page.west) {%
    \begin{minipage}{0.295\paperwidth}
      \centering
      {\color{PRI}\rule{0.22\paperwidth}{1.2pt}}\\[6pt]
      {\color{darkGray}\scriptsize\bfseries\MakeUppercase{Presented by}}\\[9pt]
      {\color{PRI}\small\bfseries Farhan Labib}\\
      {\color{medGray}\scriptsize\texttt{2205071}}\\[8pt]
      {\color{PRI}\small\bfseries Fatima Sad Sudipta}\\
      {\color{medGray}\scriptsize\texttt{2205063}}\\[9pt]
      {\color{PRI}\rule{0.22\paperwidth}{1.2pt}}\\[5pt]
      {\color{medGray}\tiny
        Bangladesh University of\\Engineering \& Technology}
    \end{minipage}%
  };

\end{tikzpicture}
\end{frame}

\begin{frame}{Outline}
  \begin{tikzpicture}[remember picture, overlay]
    \node[
      anchor=east,
      opacity=0.08,
      inner sep=0pt
    ] at ([xshift=0.01\paperwidth, yshift=-0.05\paperheight] current page.east)
      {\includegraphics[width=0.50\paperwidth]{buet_logo.png}};
  \end{tikzpicture}
  \vspace*{-0.9em}
  {\setbeamertemplate{section in toc}{%
     \leavevmode\leftskip=0.5em$\bullet$\hspace{0.5em}\inserttocsection\par\vspace{0.15em}}%
   \tableofcontents[hideallsubsections]}
\end{frame}

% =============================================================
\section{Introduction}
% =============================================================

\begin{frame}{The Problem}
  \begin{columns}[T]
    \column{0.46\textwidth}
      \begin{block}{Traditional Indexes}
        \begin{itemize}
          \item B-trees, hash tables
          \item One-dimensional keys
          \item Exact-match queries
        \end{itemize}
      \end{block}
      \vspace{0.4em}
      \begin{alertblock}<2->{They struggle with\ldots}
        \begin{itemize}
          \item Multi-dimensional data
          \item Objects with \emph{spatial extent}
          \item ``Find all objects overlapping $S$''
        \end{itemize}
      \end{alertblock}
    \column{0.46\textwidth}
      \begin{exampleblock}<3->{Real-world examples}
        \begin{itemize}
          \item GIS: counties, roads, buildings
          \item CAD: circuit layouts
          \item Games: collision detection
          \item Maps: nearest-place search
        \end{itemize}
      \end{exampleblock}
      \vspace{0.4em}
      \begin{block}<3->{}
        \centering\alert{\bfseries We need a better structure!}
      \end{block}
  \end{columns}
\end{frame}

% =============================================================
\section{R-Tree Structure}
% =============================================================

\begin{frame}{Core Idea: Minimum Bounding Rectangles}
  \begin{columns}[c]
    \column{0.50\textwidth}
    \begin{center}
    \begin{tikzpicture}[scale=0.70]
      \spatialgrid{7}{4.8}
      % Objects — always solid
      \fill[SEC!35, draw=SEC, thick, rounded corners=1pt] (0.4,0.6) rectangle (1.9,1.6);
      \node[font=\tiny\bfseries] at(1.15,1.1){A};
      \fill[SEC!35, draw=SEC, thick, rounded corners=1pt] (0.7,2.4) rectangle (2.1,3.6);
      \node[font=\tiny\bfseries] at(1.4,3.0){B};
      \fill[SEC!35, draw=SEC, thick, rounded corners=1pt] (3.4,0.4) rectangle (4.8,1.7);
      \node[font=\tiny\bfseries] at(4.1,1.05){C};
      \fill[SEC!35, draw=SEC, thick, rounded corners=1pt] (3.8,2.5) rectangle (6.0,3.8);
      \node[font=\tiny\bfseries] at(4.9,3.15){D};
      % MBR 1 — step 2
      \begin{scope}
        \only<1>{\tikzset{every path/.style={ghost}}}
        \draw[PRI, very thick, dashed, rounded corners=2pt]
              (0.25,0.45) rectangle (2.25,3.75);
        \node[PRI, font=\small\bfseries] at(1.25,4.05){MBR$_1$};
      \end{scope}
      % MBR 2 — step 3
      \begin{scope}
        \only<1-2>{\tikzset{every path/.style={ghost}}}
        \draw[HL!80!black, very thick, dashed, rounded corners=2pt]
              (3.25,0.25) rectangle (6.15,3.95);
        \node[HL!80!black, font=\small\bfseries] at(4.7,4.25){MBR$_2$};
      \end{scope}
    \end{tikzpicture}
    \end{center}
    \column{0.46\textwidth}
      \begin{block}{Step \only<1>{1 — Objects}\only<2>{2 — Group left pair}\only<3->{3 — Group right pair}}
        \only<1>{Four spatial objects in 2-D space.}%
        \only<2>{Draw the tightest rectangle around A and B: the \alert{Minimum Bounding Rectangle (MBR)}.}%
        \only<3->{Each MBR becomes one \alert{entry} in an R-tree internal node. MBRs \emph{may} overlap.}%
      \end{block}
      \vspace{0.4em}
      \begin{itemize}
        \item<2-> Every tree node $\approx$ one disk page
        \item<2-> $m \le$ entries $\le M$ per node
        \item<3-> All leaves at the same depth
      \end{itemize}
  \end{columns}
\end{frame}

\begin{frame}{R-Tree Node Types \& Tree Shape}
  \begin{columns}[T]
    \column{0.44\textwidth}
      \begin{block}{Leaf node entry\quad $(I,\ tuple\text{-}id)$}
        \begin{itemize}
          \item $I$ = MBR tightly wrapping the object
          \item $tuple\text{-}id$ = pointer to data record
        \end{itemize}
      \end{block}
      \vspace{0.3em}
      \begin{block}{Internal node entry\quad $(I,\ child\text{-}ptr)$}
        \begin{itemize}
          \item $I$ = MBR covering \emph{all} children's MBRs
          \item $child\text{-}ptr$ = address of child page
        \end{itemize}
      \end{block}
    \column{0.52\textwidth}
    \begin{center}
    \begin{tikzpicture}[
      level distance=1.15cm,
      level 1/.style={sibling distance=4.0cm},
      level 2/.style={sibling distance=1.7cm},
    ]
      \node[rootnode] {Root}
        child {node[treenode] {MBR$_1$}
          child {node[leafnode, font=\tiny\bfseries]{A}}
          child {node[leafnode, font=\tiny\bfseries]{B}}
        }
        child {node[treenode] {MBR$_2$}
          child {node[leafnode, font=\tiny\bfseries]{C}}
          child {node[leafnode, font=\tiny\bfseries]{D}}
        };
    \end{tikzpicture}
    \end{center}
  \end{columns}
\end{frame}

% =============================================================
\section{Search Simulation}
% =============================================================

%% ── SEARCH SLIDE 1 of 4  (ghost future elements) ────────────
\begin{frame}{Simulation: Search — Step 1, Tree Structure}
  \begin{columns}[c]
    \column{0.56\textwidth}
    \begin{center}
    \begin{tikzpicture}[scale=0.66]
      \spatialgrid{8.8}{5.2}
      % objects
      \fill[blue!22,  draw=blue!55, thick] (0.5,0.9)  rectangle (1.9,2.0); \node[font=\tiny] at(1.2,1.45){Obj1};
      \fill[blue!22,  draw=blue!55, thick] (0.8,3.0)  rectangle (2.2,4.2); \node[font=\tiny] at(1.5,3.6){Obj2};
      \fill[green!25, draw=green!60!black, thick] (3.4,0.7)  rectangle (4.8,2.0); \node[font=\tiny] at(4.1,1.35){Obj3};
      \fill[green!25, draw=green!60!black, thick] (3.9,2.8)  rectangle (5.6,4.2); \node[font=\tiny] at(4.75,3.5){Obj4};
      \fill[orange!25,draw=orange!75, thick] (6.4,0.9)  rectangle (8.2,2.6); \node[font=\tiny] at(7.3,1.75){Obj5};
      \fill[orange!25,draw=orange!75, thick] (6.8,3.3)  rectangle (8.4,4.7); \node[font=\tiny] at(7.6,4.0){Obj6};
      % leaf MBRs — SOLID on step 1
      \draw[blue, mbr, thick, dashed]          (0.4,0.8) rectangle (2.3,4.3);
      \node[blue, font=\small\bfseries]        at(1.35,4.5){R8};
      \draw[green!60!black, mbr, thick, dashed](3.3,0.6) rectangle (5.7,4.3);
      \node[green!60!black, font=\small\bfseries] at(4.5,4.5){R9};
      \draw[orange!75, mbr, thick, dashed]     (6.3,0.8) rectangle (8.5,4.8);
      \node[orange!75, font=\small\bfseries]   at(7.4,5.0){R10};
      % query box — GHOST on step 1
      \draw[PRI, very thick, ghost] (1.4,1.8) rectangle (5.2,4.8);
    \end{tikzpicture}
    \end{center}
    \column{0.40\textwidth}
      \begin{block}{\textbf{Step 1} — Tree structure}
        Three leaf groups: \textbf{R8}, \textbf{R9}, \textbf{R10}.\\[0.3em]
        Each group's MBR tightly wraps its objects.
      \end{block}
      \vspace{0.4em}
      \begin{block}<1->[ghost]{Step 2 — Coming next}
        \textcolor{gray!60}{\small Issue a spatial query box $S$\ldots}
      \end{block}
  \end{columns}
\end{frame}

%% ── SEARCH SLIDE 2 of 4 ─────────────────────────────────────
\begin{frame}{Simulation: Search — Step 2, Issue Query $S$}
  \begin{columns}[c]
    \column{0.56\textwidth}
    \begin{center}
    \begin{tikzpicture}[scale=0.66]
      \spatialgrid{8.8}{5.2}
      \fill[blue!22,  draw=blue!55, thick] (0.5,0.9)  rectangle (1.9,2.0); \node[font=\tiny] at(1.2,1.45){Obj1};
      \fill[blue!22,  draw=blue!55, thick] (0.8,3.0)  rectangle (2.2,4.2); \node[font=\tiny] at(1.5,3.6){Obj2};
      \fill[green!25, draw=green!60!black, thick] (3.4,0.7)  rectangle (4.8,2.0); \node[font=\tiny] at(4.1,1.35){Obj3};
      \fill[green!25, draw=green!60!black, thick] (3.9,2.8)  rectangle (5.6,4.2); \node[font=\tiny] at(4.75,3.5){Obj4};
      \fill[orange!25,draw=orange!75, thick] (6.4,0.9)  rectangle (8.2,2.6); \node[font=\tiny] at(7.3,1.75){Obj5};
      \fill[orange!25,draw=orange!75, thick] (6.8,3.3)  rectangle (8.4,4.7); \node[font=\tiny] at(7.6,4.0){Obj6};
      % MBRs — normal
      \draw[blue, mbr, thick, dashed]          (0.4,0.8) rectangle (2.3,4.3);
      \node[blue, font=\small\bfseries]        at(1.00,4.5){R8};
      \draw[green!60!black, mbr, thick, dashed](3.3,0.6) rectangle (5.7,4.3);
      \node[green!60!black, font=\small\bfseries] at(4.5,4.5){R9};
      \draw[orange!75, mbr, thick, dashed]     (6.3,0.8) rectangle (8.5,4.8);
      \node[orange!75, font=\small\bfseries]   at(7.4,5.0){R10};
      % query box — NOW SOLID
      \draw[PRI, very thick] (1.4,1.8) rectangle (5.2,4.8);
      \node[PRI, font=\small\bfseries] at(3.3,5.15){$S$ (query)};
      % ghost: highlight coming next
      \draw[PRI, very thick, ghost] (0.4,0.8) rectangle (2.3,4.3);
      \draw[PRI, very thick, ghost] (3.3,0.6) rectangle (5.7,4.3);
    \end{tikzpicture}
    \end{center}
    \column{0.40\textwidth}
      \begin{block}{Step 2 — Issue query $S$}
        For each MBR, ask:\\
        \textbf{Does it overlap $S$?}
      \end{block}
      \vspace{0.4em}
      \begin{alertblock}<1->{Step 3 — Coming next}
        \textcolor{gray!60}{\small R8 and R9 will match\ldots}
      \end{alertblock}
  \end{columns}
\end{frame}

%% ── SEARCH SLIDE 3 of 4 ─────────────────────────────────────
\begin{frame}{Simulation: Search — Step 3, R8 \& R9 Match}
  \begin{columns}[c]
    \column{0.56\textwidth}
    \begin{center}
    \begin{tikzpicture}[scale=0.66]
      \spatialgrid{8.8}{5.2}
      \fill[blue!22,  draw=blue!55, thick] (0.5,0.9)  rectangle (1.9,2.0); \node[font=\tiny] at(1.2,1.45){Obj1};
      \fill[blue!22,  draw=blue!55, thick] (0.8,3.0)  rectangle (2.2,4.2); \node[font=\tiny] at(1.5,3.6){Obj2};
      \fill[green!25, draw=green!60!black, thick] (3.4,0.7)  rectangle (4.8,2.0); \node[font=\tiny] at(4.1,1.35){Obj3};
      \fill[green!25, draw=green!60!black, thick] (3.9,2.8)  rectangle (5.6,4.2); \node[font=\tiny] at(4.75,3.5){Obj4};
      % R10 objects — ghost
      \fill[orange!10, draw=orange!30, thick, ghost] (6.4,0.9) rectangle (8.2,2.6);
      \fill[orange!10, draw=orange!30, thick, ghost] (6.8,3.3) rectangle (8.4,4.7);
      % query
      \draw[PRI, very thick] (1.4,1.8) rectangle (5.2,4.8);
      \node[PRI, font=\small\bfseries] at(3.3,5.15){$S$ (query)};
      % R8 and R9 highlighted
      \fill[PRI!12] (0.4,0.8) rectangle (2.3,4.3);
      \draw[PRI, very thick] (0.4,0.8) rectangle (2.3,4.3);
      \node[PRI, font=\small\bfseries] at(1.35,4.5){R8 \checkmark};
      \fill[PRI!12] (3.3,0.6) rectangle (5.7,4.3);
      \draw[PRI, very thick] (3.3,0.6) rectangle (5.7,4.3);
      \node[PRI, font=\small\bfseries] at(4.5,4.5){R9 \checkmark};
      % R10 MBR — ghost
      \draw[orange!30, mbr, thick, dashed, ghost] (6.3,0.8) rectangle (8.5,4.8);
      \node[orange!30, font=\small\bfseries, ghost] at(7.4,5.0){R10};
      % ghost cross for next step
      \draw[PRI!30, very thick, ghost] (6.3,0.8) -- (8.5,4.8);
      \draw[PRI!30, very thick, ghost] (6.3,4.8) -- (8.5,0.8);
    \end{tikzpicture}
    \end{center}
    \column{0.40\textwidth}
      \begin{exampleblock}{Step 3 — R8 and R9 match}
        Both MBRs overlap $S$.\\
        \textbf{Descend into both} subtrees and report overlapping objects.
      \end{exampleblock}
      \vspace{0.4em}
      \begin{alertblock}<1->{Step 4 — Coming next}
        \textcolor{gray!60}{\small R10 will be pruned entirely\ldots}
      \end{alertblock}
  \end{columns}
\end{frame}

%% ── SEARCH SLIDE 4 of 4 ─────────────────────────────────────
\begin{frame}{Simulation: Search — Step 4, R10 Pruned}
  \begin{columns}[c]
    \column{0.56\textwidth}
    \begin{center}
    \begin{tikzpicture}[scale=0.66]
      \spatialgrid{8.8}{5.2}
      \fill[blue!22,  draw=blue!55, thick] (0.5,0.9)  rectangle (1.9,2.0); \node[font=\tiny] at(1.2,1.45){Obj1};
      \fill[blue!22,  draw=blue!55, thick] (0.8,3.0)  rectangle (2.2,4.2); \node[font=\tiny] at(1.5,3.6){Obj2};
      \fill[green!25, draw=green!60!black, thick] (3.4,0.7)  rectangle (4.8,2.0); \node[font=\tiny] at(4.1,1.35){Obj3};
      \fill[green!25, draw=green!60!black, thick] (3.9,2.8)  rectangle (5.6,4.2); \node[font=\tiny] at(4.75,3.5){Obj4};
      % query
      \draw[PRI, very thick] (1.4,1.8) rectangle (5.2,4.8);
      \node[PRI, font=\small\bfseries] at(3.3,5.15){$S$};
      % R8 and R9 highlighted
      \fill[PRI!12] (0.2,0.6) rectangle (2.5,4.5);
      \draw[PRI, very thick] (0.2,0.6) rectangle (2.5,4.5);
      \node[PRI, font=\small\bfseries] at(1.35,4.85){R8 \checkmark};
      \fill[PRI!12] (3.1,0.4) rectangle (5.9,4.5);
      \draw[PRI, very thick] (3.1,0.4) rectangle (5.9,4.5);
      \node[PRI, font=\small\bfseries] at(4.5,4.85){R9 \checkmark};
      % R10 — greyed out and crossed
      \fill[white, opacity=0.7] (6.3,0.8) rectangle (8.5,4.8);
      \fill[gray!8]             (6.3,0.8) rectangle (8.5,4.8);
      \draw[gray!40, mbr, thick, dashed] (6.3,0.8) rectangle (8.5,4.8);
      \draw[PRI!70, very thick, line cap=round] (6.3,0.8) -- (8.5,4.8);
      \draw[PRI!70, very thick, line cap=round] (6.3,4.8) -- (8.5,0.8);
      \node[PRI!80, font=\small\bfseries] at(7.4,5.0){R10 PRUNED};
    \end{tikzpicture}
    \end{center}
    \column{0.40\textwidth}
      \begin{exampleblock}{Step 4 — R10 pruned}
        $R10 \cap S = \emptyset$\\[0.3em]
        Skip the \textbf{entire} R10 subtree.\\
        \textbf{Zero extra disk reads!}
      \end{exampleblock}
      \vspace{0.5em}
      \begin{block}{Key lesson}
        Good MBRs $\Rightarrow$ more pruning\\
        $\Rightarrow$ fewer disk reads\\
        $\Rightarrow$ \alert{faster queries}
      \end{block}
  \end{columns}
\end{frame}

% =============================================================
\section{Insertion \& Node Splitting}
% =============================================================

\begin{frame}{Insertion — Overview}
  \begin{columns}[T]
    \column{0.52\textwidth}
    \begin{enumerate}
      \item<1-> \textbf{ChooseLeaf} — pick child needing \alert{least MBR enlargement}
      \item<2-> \textbf{Insert} entry into chosen leaf
      \item<3-> If leaf has $> M$ entries: \alert{SplitNode}
      \item<4-> \textbf{AdjustTree} — update ancestor MBRs upward
      \item<5-> If root split: create \alert{new root} one level higher
    \end{enumerate}
    \column{0.44\textwidth}
      \begin{alertblock}<3->{The critical step}
        Quality of \textbf{node splitting} drives future search performance.\\[0.2em]
        Goal: small, non-overlapping MBRs.
      \end{alertblock}
  \end{columns}
\end{frame}

%% ── CHOOSELEAF: 3 steps ─────────────────────────────────────
\begin{frame}{Simulation: ChooseLeaf — Step 1, New Object Arrives}
  \begin{columns}[c]
    \column{0.54\textwidth}
    \begin{center}
    \begin{tikzpicture}[scale=0.64]
      \spatialgrid{8.5}{5.0}
      \fill[blue!22, draw=blue!55, thick]           (0.4,0.8) rectangle (1.8,2.0);
      \fill[blue!22, draw=blue!55, thick]           (0.7,3.1) rectangle (2.1,4.2);
      \fill[green!25,draw=green!60!black, thick]    (3.3,0.7) rectangle (4.7,2.0);
      \fill[green!25,draw=green!60!black, thick]    (3.8,2.9) rectangle (5.5,4.1);
      \fill[orange!25,draw=orange!75, thick]        (6.3,0.8) rectangle (8.0,2.5);
      \fill[orange!25,draw=orange!75, thick]        (6.7,3.2) rectangle (8.2,4.5);
      % existing MBRs
      \draw[blue, mbr, thick, dashed]             (0.3,0.7) rectangle (2.2,4.3);
      \node[blue,font=\tiny\bfseries]             at(1.25,4.5){R8};
      \draw[green!60!black, mbr, thick, dashed]   (3.2,0.6) rectangle (5.6,4.2);
      \node[green!60!black,font=\tiny\bfseries]   at(4.4,4.4){R9};
      \draw[orange!75, mbr, thick, dashed]        (6.2,0.7) rectangle (8.3,4.6);
      \node[orange!75,font=\tiny\bfseries]        at(7.25,4.8){R10};
      % NEW object
      \fill[PRI!70, draw=PRI, very thick, rounded corners=1pt] (3.6,4.6) rectangle (5.0,5.2);
      \node[white, font=\tiny\bfseries] at(4.3,4.9){Obj7 NEW};
      % ghost arrows
      \draw[blue,ghost,thick,->]           (1.25,4.3) -- (3.5,5.1);
      \draw[green!60!black,ghost,thick,->] (4.3,4.22) -- (3.65,4.6);
      \draw[orange!75,ghost,thick,->]      (7.25,4.6) -- (5.1,5.1);
    \end{tikzpicture}
    \end{center}
    \column{0.42\textwidth}
      \begin{block}{Step 1 — Obj7 arrives}
        Must find the best leaf node to insert into.
      \end{block}
      \vspace{0.3em}
      \begin{alertblock}<1->{Step 2 — Coming next}
        \textcolor{gray!60}{\small Compute enlargements for each group\ldots}
      \end{alertblock}
  \end{columns}
\end{frame}

\begin{frame}{Simulation: ChooseLeaf — Step 2, Compare Enlargements}
  \begin{columns}[c]
    \column{0.54\textwidth}
    \begin{center}
    \begin{tikzpicture}[scale=0.64]
      \spatialgrid{8.5}{5.4}
      \fill[blue!22, draw=blue!55, thick]           (0.4,0.8) rectangle (1.8,2.0);
      \fill[blue!22, draw=blue!55, thick]           (0.7,3.1) rectangle (2.1,4.2);
      \fill[green!25,draw=green!60!black, thick]    (3.3,0.7) rectangle (4.7,2.0);
      \fill[green!25,draw=green!60!black, thick]    (3.8,2.9) rectangle (5.5,4.1);
      \fill[orange!25,draw=orange!75, thick]        (6.3,0.8) rectangle (8.0,2.5);
      \fill[orange!25,draw=orange!75, thick]        (6.7,3.2) rectangle (8.2,4.5);
      \draw[blue, mbr, thick, dashed]             (0.3,0.7) rectangle (2.2,4.3);
      \node[blue,font=\tiny\bfseries]             at(1.25,4.5){R8};
      \draw[green!60!black, mbr, thick, dashed]   (3.2,0.6) rectangle (5.6,4.2);
      \node[green!60!black,font=\tiny\bfseries]   at(4.4,4.4){R9};
      \draw[orange!75, mbr, thick, dashed]        (6.2,0.7) rectangle (8.3,4.6);
      \node[orange!75,font=\tiny\bfseries]        at(7.25,4.8){R10};
      \fill[PRI!70, draw=PRI, very thick, rounded corners=1pt] (3.6,4.6) rectangle (5.0,5.2);
      \node[white, font=\tiny\bfseries] at(4.3,4.9){Obj7};
      % enlargement arrows — SOLID
      \draw[blue,thick,->]           (1.25,4.3) -- (3.5,5.1);
      \node[blue,font=\tiny,align=center] at(0.5,4.9){R8\\{\scriptsize big}};
      \draw[green!60!black,very thick,->] (4.3,4.22) -- (3.65,4.6);
      \node[green!60!black,font=\footnotesize\bfseries] at(4.4,0.25){\checkmark min};
      \draw[orange!75,thick,->]      (7.25,4.6) -- (5.1,5.1);
      \node[orange!75,font=\tiny,align=center] at(8.1,4.85){R10\\{\scriptsize big}};
    \end{tikzpicture}
    \end{center}
    \column{0.42\textwidth}
      \begin{block}{Step 2 — Enlargements}
        R8: large growth needed\\
        \alert{R9: minimal growth} \checkmark\\
        R10: large growth needed
      \end{block}
      \vspace{0.3em}
      \begin{alertblock}<1->{Step 3 — Coming next}
        \textcolor{gray!60}{\small Insert into R9 and update MBR\ldots}
      \end{alertblock}
  \end{columns}
\end{frame}

\begin{frame}{Simulation: ChooseLeaf — Step 3, Insert into R9}
  \begin{columns}[c]
    \column{0.54\textwidth}
    \begin{center}
    \begin{tikzpicture}[scale=0.64]
      \spatialgrid{8.5}{5.4}
      \fill[blue!22, draw=blue!55, thick]           (0.4,0.8) rectangle (1.8,2.0);
      \fill[blue!22, draw=blue!55, thick]           (0.7,3.1) rectangle (2.1,4.2);
      \fill[green!25,draw=green!60!black, thick]    (3.3,0.7) rectangle (4.7,2.0);
      \fill[green!25,draw=green!60!black, thick]    (3.8,2.9) rectangle (5.5,4.1);
      \fill[orange!25,draw=orange!75, thick]        (6.3,0.8) rectangle (8.0,2.5);
      \fill[orange!25,draw=orange!75, thick]        (6.7,3.2) rectangle (8.2,4.5);
      \draw[blue, mbr, thick, dashed]             (0.3,0.7) rectangle (2.2,4.3);
      \node[blue,font=\tiny\bfseries]             at(1.25,4.5){R8};
      \draw[orange!75, mbr, thick, dashed]        (6.2,0.7) rectangle (8.3,4.6);
      \node[orange!75,font=\tiny\bfseries]        at(7.25,4.8){R10};
      % Obj7 inserted
      \fill[PRI!70, draw=PRI, very thick, rounded corners=1pt] (3.6,4.6) rectangle (5.0,5.2);
      \node[white, font=\tiny\bfseries] at(4.3,4.9){Obj7};
      % EXPANDED R9 MBR
      \draw[green!60!black, very thick] (3.2,0.6) rectangle (5.6,5.3);
      \node[green!60!black,font=\small\bfseries] at(4.4,5.6){R9 (expanded)};
    \end{tikzpicture}
    \end{center}
    \column{0.42\textwidth}
      \begin{exampleblock}{Step 3 — Inserted!}
        R9 = \{Obj3, Obj4, \textbf{Obj7}\}\\
        3 entries $= M$. No split.\\[0.3em]
        MBR updated to wrap Obj7.
      \end{exampleblock}
  \end{columns}
\end{frame}

% =============================================================
%  QUAD SPLIT DEEP DIVE
% =============================================================

\begin{frame}{Node Splitting — Three Algorithms}
  \begin{columns}[T]
    \column{0.3\textwidth}
      \begin{block}{Exhaustive}
        All $2^M$ groupings.\\
        Optimal quality.\\
        $O(2^M)$ — too slow.
      \end{block}
    \column{0.3\textwidth}
      \begin{alertblock}{Quadratic \textit{(focus)}}
        Heuristic seeds.\\
        Good quality.\\
        $O(M^2)$ per split.
      \end{alertblock}
    \column{0.3\textwidth}
      \begin{block}{Linear}
        Axis extremes.\\
        Acceptable quality.\\
        $O(M)$ — fastest.
      \end{block}
  \end{columns}
  \vspace{0.6em}
  \begin{block}{Goal of any split}
    Two nodes with \alert{small, minimally overlapping} MBRs.\\
    Better split $\Rightarrow$ better search pruning $\Rightarrow$ fewer disk reads.
  \end{block}
\end{frame}

%% ── PICKSEED step 1 ─────────────────────────────────────────
\begin{frame}{Quadratic Split — Step 1: PickSeeds, Meet the Entries}
  \begin{columns}[c]
    \column{0.50\textwidth}
    \begin{center}
    \begin{tikzpicture}[scale=0.82]
      \draw[LG!70, very thin, step=1] (0,0) grid (7,5.4);
      \draw[-{Stealth[scale=0.7]}](0,0)--(7.3,0) node[right,font=\tiny]{$x$};
      \draw[-{Stealth[scale=0.7]}](0,0)--(0,5.6) node[above,font=\tiny]{$y$};
      % 4 entries
      \fill[blue!28,  draw=blue!70,  thick, rounded corners=1pt] (0.3,3.4) rectangle (2.0,5.0);
      \node[font=\normalsize\bfseries] at(1.15,4.2){A};
      \fill[green!28, draw=green!65!black, thick, rounded corners=1pt] (0.3,0.3) rectangle (2.0,1.8);
      \node[font=\normalsize\bfseries] at(1.15,1.05){B};
      \fill[orange!32,draw=orange!80!black, thick, rounded corners=1pt] (4.8,3.4) rectangle (6.7,5.1);
      \node[font=\normalsize\bfseries] at(5.75,4.25){C};
      \fill[purple!28,draw=purple!70, thick, rounded corners=1pt] (4.8,0.3) rectangle (6.7,1.8);
      \node[font=\normalsize\bfseries] at(5.75,1.05){D};
      % ghost: worst-pair MBR coming next
      \draw[PRI, very thick, dashed, ghost] (0.1,0.1) rectangle (6.9,5.3);
    \end{tikzpicture}
    \end{center}
    \column{0.46\textwidth}
      \begin{block}{Overflowing node}
        4 entries: A, B, C, D ($> M = 3$).\\[0.3em]
        Must split into two groups.
      \end{block}
      \vspace{0.3em}
      \begin{block}{PickSeeds formula}
        For every pair $(E_i, E_j)$:
        {\small\[d = \text{area}(E_i \cup E_j) - \text{area}(E_i) - \text{area}(E_j)\]}
        Choose pair with \alert{largest $d$} (most wasted space).
      \end{block}
  \end{columns}
\end{frame}

%% ── PICKSEED step 2 ─────────────────────────────────────────
\begin{frame}{Quadratic Split — Step 2: PickSeeds, Worst Pair Found}
  \begin{columns}[c]
    \column{0.50\textwidth}
    \begin{center}
    \begin{tikzpicture}[scale=0.82]
      \draw[LG!70, very thin, step=1] (0,0) grid (7,5.4);
      \draw[-{Stealth[scale=0.7]}](0,0)--(7.3,0) node[right,font=\tiny]{$x$};
      \draw[-{Stealth[scale=0.7]}](0,0)--(0,5.6) node[above,font=\tiny]{$y$};
      % non-seed entries ghost
      \fill[green!12, draw=green!35, thick, rounded corners=1pt, ghost] (0.3,0.3) rectangle (2.0,1.8);
      \node[font=\normalsize\bfseries, ghost] at(1.15,1.05){B};
      \fill[orange!14,draw=orange!35, thick, rounded corners=1pt, ghost] (4.8,3.4) rectangle (6.7,5.1);
      \node[font=\normalsize\bfseries, ghost] at(5.75,4.25){C};
      % seeds SOLID + highlighted
      \fill[blue!28,  draw=PRI, very thick, rounded corners=1pt] (0.3,3.4) rectangle (2.0,5.0);
      \node[font=\normalsize\bfseries, PRI] at(1.15,4.2){\textbf{A}};
      \fill[purple!28,draw=PRI, very thick, rounded corners=1pt] (4.8,0.3) rectangle (6.7,1.8);
      \node[font=\normalsize\bfseries, PRI] at(5.75,1.05){\textbf{D}};
      % worst-pair MBR
      \fill[PRI!8] (0.1,0.1) rectangle (6.9,5.3);
      \draw[PRI, very thick, dashed, rounded corners=2pt] (0.1,0.1) rectangle (6.9,5.3);
      \node[PRI, font=\small\bfseries] at(3.5,-0.4){MBR$(A \cup D)$ = max wasted area};
    \end{tikzpicture}
    \end{center}
    \column{0.46\textwidth}
      \begin{exampleblock}{Seeds: A and D}
        Diagonally opposite — their joint MBR wastes the most space.\\[0.3em]
        $s_1 = A \to G_1$\\
        $s_2 = D \to G_2$
      \end{exampleblock}
      \vspace{0.3em}
      \begin{alertblock}<1->{Next — PickNext}
        \textcolor{gray!60}{\small Assign B and C to their preferred group\ldots}
      \end{alertblock}
  \end{columns}
\end{frame}

%% ── PICKNEXT ────────────────────────────────────────────────
\begin{frame}{Quadratic Split — Step 3: PickNext, Assign Remaining}
  \begin{columns}[c]
    \column{0.50\textwidth}
    \begin{center}
    \begin{tikzpicture}[scale=0.82]
      \draw[LG!70, very thin, step=1] (0,0) grid (7,5.4);
      \draw[-{Stealth[scale=0.7]}](0,0)--(7.3,0) node[right,font=\tiny]{$x$};
      \draw[-{Stealth[scale=0.7]}](0,0)--(0,5.6) node[above,font=\tiny]{$y$};
      % all 4 entries
      \fill[blue!28,  draw=blue!70,  thick, rounded corners=1pt] (0.3,3.4) rectangle (2.0,5.0);
      \node[font=\normalsize\bfseries] at(1.15,4.2){A};
      \fill[green!28, draw=green!65!black, thick, rounded corners=1pt] (0.3,0.3) rectangle (2.0,1.8);
      \node[font=\normalsize\bfseries] at(1.15,1.05){B};
      \fill[orange!32,draw=orange!80!black, thick, rounded corners=1pt] (4.8,3.4) rectangle (6.7,5.1);
      \node[font=\normalsize\bfseries] at(5.75,4.25){C};
      \fill[purple!28,draw=purple!70, thick, rounded corners=1pt] (4.8,0.3) rectangle (6.7,1.8);
      \node[font=\normalsize\bfseries] at(5.75,1.05){D};
      % G1 MBR (A only so far)
      \draw[blue!80!black, thick, dashed] (0.1,3.2) rectangle (2.2,5.2);
      \node[blue!80!black,font=\tiny\bfseries] at(1.15,5.45){$G_1$};
      % G2 MBR (D only)
      \draw[purple!80, thick, dashed] (4.6,0.1) rectangle (6.9,2.0);
      \node[purple!80,font=\tiny\bfseries] at(5.75,-0.3){$G_2$};
      % preference arrows
      \draw[-{Stealth}, green!65!black, very thick] (1.15,1.8) -- (1.15,3.2)
            node[midway, right, font=\tiny]{B $\to G_1$};
      \draw[-{Stealth}, orange!80!black, very thick] (5.75,3.4) -- (5.75,2.0)
            node[midway, right, font=\tiny]{C $\to G_2$};
    \end{tikzpicture}
    \end{center}
    \column{0.46\textwidth}
      \begin{block}{Rule}
        $\delta_i = \Delta\text{area}(G_1) - \Delta\text{area}(G_2)$\\[0.2em]
        Assign entry with \alert{largest $|\delta_i|$} first.
      \end{block}
      \vspace{0.3em}
      \begin{exampleblock}{Round 1 result}
        B strongly prefers $G_1$ (near A).\\
        C strongly prefers $G_2$ (near D).\\[0.2em]
        Both assigned. \textbf{Split complete!}
      \end{exampleblock}
  \end{columns}
\end{frame}

%% ── FINAL GROUPS ────────────────────────────────────────────
\begin{frame}{Quadratic Split — Step 4: Final Groups}
  \begin{columns}[c]
    \column{0.50\textwidth}
    \begin{center}
    \begin{tikzpicture}[scale=0.82]
      \draw[LG!70, very thin, step=1] (0,0) grid (7,5.4);
      \draw[-{Stealth[scale=0.7]}](0,0)--(7.3,0) node[right,font=\tiny]{$x$};
      \draw[-{Stealth[scale=0.7]}](0,0)--(0,5.6) node[above,font=\tiny]{$y$};
      \fill[blue!28,  draw=blue!70,  thick, rounded corners=1pt] (0.3,3.4) rectangle (2.0,5.0);
      \node[font=\normalsize\bfseries] at(1.15,4.2){A};
      \fill[green!28, draw=green!65!black, thick, rounded corners=1pt] (0.3,0.3) rectangle (2.0,1.8);
      \node[font=\normalsize\bfseries] at(1.15,1.05){B};
      \fill[orange!32,draw=orange!80!black, thick, rounded corners=1pt] (4.8,3.4) rectangle (6.7,5.1);
      \node[font=\normalsize\bfseries] at(5.75,4.25){C};
      \fill[purple!28,draw=purple!70, thick, rounded corners=1pt] (4.8,0.3) rectangle (6.7,1.8);
      \node[font=\normalsize\bfseries] at(5.75,1.05){D};
      % tight final MBRs
      \draw[blue!80!black, very thick, rounded corners=2pt] (0.1,0.1) rectangle (2.2,5.2);
      \node[blue!80!black,font=\small\bfseries] at(1.15,-0.45){Node 1};
      \draw[purple!80!black, very thick, rounded corners=2pt] (4.6,0.1) rectangle (6.9,5.3);
      \node[purple!80!black,font=\small\bfseries] at(5.75,-0.45){Node 2};
    \end{tikzpicture}
    \end{center}
    \column{0.46\textwidth}
      \begin{exampleblock}{Result}
        \textbf{Node 1}: \{A, B\} — left cluster\\
        \textbf{Node 2}: \{C, D\} — right cluster\\[0.4em]
        Compact, \alert{non-overlapping} MBRs.
      \end{exampleblock}
      \vspace{0.3em}
      \begin{block}{AdjustTree}
        Parent gets two new entries pointing to Node 1 and Node 2.\\
        Ancestor MBRs updated upward.
      \end{block}
  \end{columns}
\end{frame}

\begin{frame}{Good Split vs Bad Split}
  \begin{center}
  \begin{tikzpicture}[scale=0.75]
    \begin{scope}
      \fill[blue!20,  draw=blue!50,  thick] (0.5,0.5) rectangle (1.5,1.5);
      \fill[blue!20,  draw=blue!50,  thick] (0.5,3.2) rectangle (1.5,4.2);
      \fill[green!20, draw=green!55!black, thick] (5.0,0.5) rectangle (6.5,1.5);
      \fill[green!20, draw=green!55!black, thick] (5.0,3.2) rectangle (6.5,4.2);
      \draw[PRI, very thick, dashed] (-0.3,0.0) rectangle (5.6,4.8);
      \draw[PRI, very thick, dashed] (3.8,0.0) rectangle (7.0,4.8);
      \fill[PRI!18, opacity=0.5] (3.8,0.0) rectangle (5.6,4.8);
      \node[PRI,font=\bfseries] at(3.35,-0.65){\alert{Bad Split}};
      \node[font=\footnotesize,align=center] at(3.35,-1.5){Large overlap\\$\Rightarrow$ query visits BOTH branches};
    \end{scope}
    \begin{scope}[xshift=9.8cm]
      \fill[blue!20,  draw=blue!50,  thick] (0.5,0.5) rectangle (1.5,1.5);
      \fill[blue!20,  draw=blue!50,  thick] (0.5,3.2) rectangle (1.5,4.2);
      \fill[green!20, draw=green!55!black, thick] (5.0,0.5) rectangle (6.5,1.5);
      \fill[green!20, draw=green!55!black, thick] (5.0,3.2) rectangle (6.5,4.2);
      \draw[blue!75!black, very thick] (-0.1,0.2) rectangle (2.1,4.6);
      \draw[green!60!black, very thick] (4.6,0.2) rectangle (7.0,4.6);
      \node[green!60!black,font=\bfseries] at(3.5,-0.65){Good Split};
      \node[font=\footnotesize,align=center] at(3.5,-1.5){No overlap\\$\Rightarrow$ query prunes one branch entirely};
    \end{scope}
  \end{tikzpicture}
  \end{center}
\end{frame}

\section{Full Insertion Walkthrough}

\begin{frame}{Walkthrough — Setup}
  R-tree with $M=3$, starting state:
  \begin{center}
  \begin{tikzpicture}[
    level distance=1.2cm,
    level 1/.style={sibling distance=3.5cm},
    level 2/.style={sibling distance=1.6cm},
  ]
    \node[rootnode] {Root}
      child {node[treenode, fill=blue!20, draw=blue!70]{R8}
        child {node[leafnode, font=\tiny\bfseries]{Obj1}}
        child {node[leafnode, font=\tiny\bfseries]{Obj2}}
      }
      child {node[treenode, fill=green!20, draw=green!60!black]{R9}
        child {node[leafnode, font=\tiny\bfseries]{Obj3}}
        child {node[leafnode, font=\tiny\bfseries]{Obj4}}
      }
      child {node[treenode, fill=orange!20, draw=orange!70]{R10}
        child {node[leafnode, font=\tiny\bfseries]{Obj5}}
        child {node[leafnode, font=\tiny\bfseries]{Obj6}}
      };
  \end{tikzpicture}
  \end{center}
  \vspace{0.3em}
  Plan: insert \textbf{Obj7} (near R9) — no overflow.\\
  Then insert \textbf{Obj8} (near R9) — triggers \alert{overflow + split}.
\end{frame}

\begin{frame}{Walkthrough — Insert Obj7, R9 Updated}
  \begin{columns}[T]
    \column{0.46\textwidth}
    \begin{center}
    \begin{tikzpicture}[scale=0.62]
      \spatialgrid{6.5}{5.0}
      \fill[blue!22, draw=blue!55, thick]        (0.3,0.7) rectangle (1.5,1.8);
      \fill[blue!22, draw=blue!55, thick]        (0.5,3.0) rectangle (1.8,4.2);
      \fill[green!25,draw=green!60!black, thick] (2.7,0.5) rectangle (3.9,1.8);  \node[font=\tiny] at(3.3,1.15){Obj3};
      \fill[green!25,draw=green!60!black, thick] (3.1,2.8) rectangle (4.7,4.0);  \node[font=\tiny] at(3.9,3.4){Obj4};
      \fill[orange!25,draw=orange!70, thick]     (5.0,0.7) rectangle (6.3,2.2);
      \fill[orange!25,draw=orange!70, thick]     (5.2,3.0) rectangle (6.3,4.1);
      \draw[blue, mbr, thick, dashed]             (0.2,0.6) rectangle (1.9,4.3);
      \draw[orange!70, mbr, thick, dashed]        (4.9,0.6) rectangle (6.4,4.2);
      % Obj7
      \fill[PRI!65, draw=PRI, thick, rounded corners=1pt] (3.2,4.4) rectangle (4.5,5.0);
      \node[white,font=\tiny\bfseries] at(3.85,4.7){Obj7};
      % Expanded R9
      \draw[green!60!black, very thick] (2.6,0.4) rectangle (4.8,5.1);
      \node[green!60!black,font=\small\bfseries] at(3.7,5.35){R9 (expanded)};
    \end{tikzpicture}
    \end{center}
    \column{0.50\textwidth}
      \begin{block}{ChooseLeaf $\to$ R9}
        R9 needs minimum enlargement.
      \end{block}
      \vspace{0.2em}
      \begin{exampleblock}{Result — No Split}
        R9 = \{Obj3, Obj4, Obj7\}\\
        3 entries $= M$. Tree unchanged.
      \end{exampleblock}
      \vspace{0.2em}
      \begin{center}
        \scalebox{0.78}{%
      \begin{tikzpicture}[
        level distance=0.9cm,
        level 1/.style={sibling distance=3.8cm},
        level 2/.style={sibling distance=1.0cm},
      ]
        \node[rootnode,font=\tiny]{Root}
          child{node[treenode,fill=blue!20,draw=blue!70,font=\tiny,minimum width=0.9cm]{R8}
            child{node[leafnode,font=\tiny,minimum width=0.8cm]{O1}}
            child{node[leafnode,font=\tiny,minimum width=0.8cm]{O2}}}
          child{node[treenode,fill=green!20,draw=green!70!black,font=\tiny,minimum width=0.9cm]{R9}
            child{node[leafnode,font=\tiny,minimum width=0.7cm]{O3}}
            child{node[leafnode,font=\tiny,minimum width=0.7cm]{O4}}
            child{node[leafnode,font=\tiny,minimum width=0.7cm,fill=PRI!35]{O7}}}
          child{node[treenode,fill=orange!20,draw=orange!70,font=\tiny,minimum width=0.9cm]{R10}
            child{node[leafnode,font=\tiny,minimum width=0.7cm]{O5}}
            child{node[leafnode,font=\tiny,minimum width=0.7cm]{O6}}};
      \end{tikzpicture}}
      \end{center}
  \end{columns}
\end{frame}

\begin{frame}{Walkthrough — Insert Obj8, Overflow!}
  \begin{columns}[T]
    \column{0.50\textwidth}
    \begin{center}
    \begin{tikzpicture}[scale=0.64]
      \spatialgrid{6.5}{5.0}
      \fill[green!25,draw=green!60!black, thick] (2.7,0.5) rectangle (3.9,1.8);  \node[font=\tiny] at(3.3,1.15){Obj3};
      \fill[green!25,draw=green!60!black, thick] (3.1,2.8) rectangle (4.7,4.0);  \node[font=\tiny] at(3.9,3.4){Obj4};
      \fill[PRI!60, draw=PRI, thick]             (3.2,4.3) rectangle (4.5,4.9);  \node[white,font=\tiny\bfseries] at(3.85,4.6){Obj7};
      % Obj8 NEW
      \fill[SEC!70, draw=SEC, very thick, rounded corners=1pt] (2.2,2.0) rectangle (3.5,3.1);
      \node[white,font=\tiny\bfseries] at(2.85,2.55){Obj8 NEW};
      \node[PRI, font=\footnotesize\bfseries] at(3.5,-0.4){R9 overflows! (4 entries)};
      % split result
      \draw[blue!80, very thick, rounded corners=2pt]   (2.0,0.3) rectangle (4.0,3.3);
      \node[blue!80,font=\tiny\bfseries] at(3.0,3.6){R9a: \{Obj3, Obj8\}};
      \draw[purple!75, very thick, rounded corners=2pt] (3.0,2.6) rectangle (4.8,5.1);
      \node[purple!75,font=\tiny\bfseries] at(3.9,5.35){R9b: \{Obj4, Obj7\}};
    \end{tikzpicture}
    \end{center}
    \column{0.46\textwidth}
      \begin{alertblock}{Overflow}
        R9 would have 4 entries $> M=3$.\\
        \textbf{SplitNode} triggered.
      \end{alertblock}
      \vspace{0.3em}
      \begin{block}{Quadratic split result}
        Seeds: Obj3 \& Obj7 (max wasted area).\\[0.2em]
        \textbf{R9a} = \{Obj3, Obj8\}\\
        \textbf{R9b} = \{Obj4, Obj7\}
      \end{block}
      \vspace{0.3em}
      \begin{exampleblock}{AdjustTree}
        Root gains R9a and R9b.\\
        4 children total — still $\le M$.
      \end{exampleblock}
  \end{columns}
\end{frame}

\begin{frame}{Walkthrough — Final Tree After Split}
  \begin{center}
  \begin{tikzpicture}[
    level distance=1.2cm,
    level 1/.style={sibling distance=3.2cm},
    level 2/.style={sibling distance=1.6cm},
  ]
    \node[rootnode] {Root}
      child {node[treenode, fill=blue!20,   draw=blue!70]   {R8}
        child {node[leafnode, font=\tiny\bfseries]{Obj1}}
        child {node[leafnode, font=\tiny\bfseries]{Obj2}}}
      child {node[treenode, fill=blue!28,   draw=blue!80]   {R9a}
        child {node[leafnode, font=\tiny\bfseries]{Obj3}}
        child {node[leafnode, font=\tiny\bfseries]{Obj8}}}
      child {node[treenode, fill=purple!20, draw=purple!70] {R9b}
        child {node[leafnode, font=\tiny\bfseries]{Obj4}}
        child {node[leafnode, font=\tiny\bfseries]{Obj7}}}
      child {node[treenode, fill=orange!20, draw=orange!70] {R10}
        child {node[leafnode, font=\tiny\bfseries]{Obj5}}
        child {node[leafnode, font=\tiny\bfseries]{Obj6}}};
  \end{tikzpicture}
  \end{center}
  \vspace{0.3em}
  Root now has 4 children. Tree height unchanged.
  All MBRs compact thanks to the quadratic split.
\end{frame}

% =============================================================
\section{Deletion}
% =============================================================

\begin{frame}{Deletion Algorithm}
  \begin{columns}[T]
    \column{0.52\textwidth}
    \begin{enumerate}
      \item<1-> \textbf{FindLeaf} — locate the leaf containing $E$
      \item<2-> \textbf{Remove} $E$ from that leaf
      \item<3-> \textbf{CondenseTree} — if node has $< m$ entries:\\
            \quad delete it; \alert{re-insert} orphaned entries
      \item<4-> Propagate MBR shrinkage upward
      \item<5-> If root has one child: shrink tree height
    \end{enumerate}
    \column{0.44\textwidth}
      \begin{block}<3->{Why re-insert, not merge?}
        \begin{itemize}
          \item Reuses Insert logic
          \item Entries find natural home
          \item Gradually \alert{improves} structure
          \item Simpler than spatial merge
        \end{itemize}
      \end{block}
  \end{columns}
\end{frame}

% =============================================================
\section{Key Takeaways}
% =============================================================

\begin{frame}{Summary}
  \begin{columns}[T]
    \column{0.48\textwidth}
      \begin{block}{What we covered}
        \begin{enumerate}
          \item MBRs group nearby spatial objects
          \item Search prunes non-overlapping subtrees
          \item ChooseLeaf: least-enlargement heuristic
          \item Quadratic split: PickSeeds $+$ PickNext
          \item Split quality drives search efficiency
          \item Deletion re-inserts orphaned entries
        \end{enumerate}
      \end{block}
    \column{0.48\textwidth}
      \begin{alertblock}{The golden rule}
        \centering\vspace{0.2em}
        \large
        Minimise \textbf{overlap} and \textbf{area} of MBRs at every split.
        \vspace{0.2em}
      \end{alertblock}
      \vspace{0.4em}
      \begin{exampleblock}{Used in practice}
        PostgreSQL/PostGIS, QGIS,\\
        ArcGIS, game engines,\\
        GEOS, Shapely, SQLite R*
      \end{exampleblock}
  \end{columns}
\end{frame}

\begin{frame}{References}
  \begin{thebibliography}{10}
  \beamertemplatearticlebibitems
  \bibitem{Guttman1984}
  Antonin Guttman
  \newblock \textit{R-Trees: A Dynamic Index Structure for Spatial Searching}
  \newblock ACM SIGMOD, 1984

  \beamertemplatearticlebibitems
  \bibitem{Wikipedia}
  Wikipedia contributors
  \newblock \textit{R-tree} ---
  \texttt{en.wikipedia.org/wiki/R-tree}
  \end{thebibliography}
\end{frame}

\begin{frame}[plain]
\begin{tikzpicture}[remember picture, overlay]

  %% White background
  \fill[white]
    (current page.south west) rectangle (current page.north east);

  %% Subtle dot grid (dark on white)
  \foreach \x in {0.5,1.5,...,15.5} {
    \foreach \y in {0.3,1.0,...,8.5} {
      \fill[darkGray, opacity=0.05] (\x,\y) circle (1pt);
    }
  }

  %% Big decorative R-tree diagram (background art, red-tinted)
  \begin{scope}[opacity=0.06, xshift=1cm, yshift=1.2cm]
    \draw[PRI, very thick] (2,4.5) -- (0.5,2.5);
    \draw[PRI, very thick] (2,4.5) -- (3.5,2.5);
    \draw[PRI, very thick] (0.5,2.5) -- (-0.5,0.8);
    \draw[PRI, very thick] (0.5,2.5) -- (1.5,0.8);
    \draw[PRI, very thick] (3.5,2.5) -- (2.5,0.8);
    \draw[PRI, very thick] (3.5,2.5) -- (4.5,0.8);
    \draw[PRI, very thick, rounded corners=4pt] (1.3,4.2) rectangle (2.7,4.9);
    \draw[PRI, very thick, rounded corners=3pt] (-0.2,2.2) rectangle (1.2,2.9);
    \draw[PRI, very thick, rounded corners=3pt] (2.8,2.2) rectangle (4.2,2.9);
    \draw[PRI, very thick, rounded corners=2pt] (-1.2,0.4) rectangle (0.2,1.1);
    \draw[PRI, very thick, rounded corners=2pt] (0.8,0.4) rectangle (2.2,1.1);
    \draw[PRI, very thick, rounded corners=2pt] (1.8,0.4) rectangle (3.2,1.1);
    \draw[PRI, very thick, rounded corners=2pt] (3.8,0.4) rectangle (5.2,1.1);
  \end{scope}

  %% Left red accent bar
  \fill[PRI]
    (current page.south west) rectangle
    ([xshift=0.012\paperwidth] current page.north west);

  %% Top red accent bar
  \fill[PRI]
    ([yshift=-0.012\paperheight] current page.north west) rectangle
    (current page.north east);

  %% Central "Thank You" — dark red on white
  \node[
    text=PRI,
    font=\Huge\bfseries,
    anchor=center,
    scale=1.6
  ] at ([yshift=0.10\paperheight] current page.center) {Thank You};

  %% Subtitle
  \node[
    text=medGray,
    font=\large,
    anchor=center
  ] at ([yshift=-0.02\paperheight] current page.center)
    {Questions \& Discussion};

  %% Separator
  \draw[PRI, line width=1.5pt]
    ([xshift=-0.25\paperwidth, yshift=0.03\paperheight] current page.center) --
    ([xshift= 0.25\paperwidth, yshift=0.03\paperheight] current page.center);

  %% Team credit strip at bottom — light red tint
  \fill[PRI, opacity=0.08]
    (current page.south west) rectangle
    ([yshift=0.14\paperheight] current page.south east);
  \draw[PRI, opacity=0.25, line width=0.6pt]
    ([yshift=0.14\paperheight] current page.south west) --
    ([yshift=0.14\paperheight] current page.south east);

  \node[text=darkGray, font=\scriptsize, anchor=west] at
    ([xshift=0.05\paperwidth, yshift=0.07\paperheight] current page.south west)
    {\textbf{Farhan Labib} \quad\texttt{2205071}
     \quad{\color{PRI}$\bullet$}\quad
     \textbf{Fatima Sad Sudipta} \quad\texttt{2205063}};

  \node[text=medGray, font=\tiny\itshape, anchor=east] at
    ([xshift=-0.05\paperwidth, yshift=0.07\paperheight] current page.south east)
    {Bangladesh University of Engineering \& Technology};

  %% Small BUET logo bottom right
  \node[anchor=south east, inner sep=8pt] at
    ([yshift=0.14\paperheight] current page.south east)
    {\includegraphics[height=0.55cm]{BUET_LOGO.png}};

\end{tikzpicture}
\end{frame}

\end{document}